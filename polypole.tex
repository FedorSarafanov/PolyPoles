\input{text/diss}

\begin{document}

\def\labauthors{Понур К.А., Сарафанов Ф.Г., Сидоров Д.А.}
\def\labgroup{420}
\def\labnumber{210}
\def\labtheme{Исследование линейных двухполюсников и четырёхполюсников}
\renewcommand{\vec}{\mathbf}
\renewcommand{\Re}{\operatorname{Re}}
\renewcommand{\Im}{\operatorname{Im}}

\input{text/titlepage}

\tableofcontents
\newpage
\section{Первая схема}
\begin{center}
	\input{chem1}
\end{center}

Рассчитаем импеданс данной схемы методом комплексных амплитуд.
\begin{equation}
	\hat{U}=U_0 e^{i(\omega t+\phi_U)}
\end{equation}
Величину $\hat{U_0}=U_0e^{i\phi_U}$ будем называть комплексной амплитудой напряжения
\begin{equation}
	I=C\diff{U}{t}
\end{equation}
Отсюда получаем:
\begin{equation}
	\hat{I}=U_0\,\omega i C\exp(i\omega t+\phi_U)
\end{equation}
И комплексная амплитуда тока:
\begin{equation}
	\hat{I_0}=U_0i\omega C e^{i\phi_U}
\end{equation}
Получаем комплексный импеданс схемы
\begin{equation}
	\hat{z}=\frac{\hat{U_0}}{\hat{I_0}}=\frac{U_0e^{i\phi_U}}{U_0i\omega C e^{i\phi_U}}=\frac{1}{i\cdot\omega C}
\end{equation}

\begin{equation}
	z=\frac{1}{\omega C}
\end{equation}
\section{Вторая схема}
\begin{center}
\documentclass[border=1pt]{standalone}
\usepackage[europeanresistors,americaninductors]{circuitikz}

\begin{document}
	
      \begin{circuitikz}[]

            \draw (0,0) to [o-] (0,2)
            to [R] (-2,2)
            to [inductor] (-4,2)
            to [-o] (-4,0)
            ; 
	\end{circuitikz}
\end{document}
\end{center}

\begin{equation}
	\hat{I}=I_0e^{i(\omega t+\phi_I)}
\end{equation}
\begin{equation}
	U=L\diff{I}{t}
\end{equation}
\begin{equation}
	\hat{U}=I_0i\omega L e^{i(\omega t +\phi_I)}
\end{equation}
Отсюда
\begin{equation}
	\hat{z}=i\omega L
	\end{equation}



\section{Третья схема}
\begin{center}
\documentclass[border=1pt]{standalone}
\usepackage[europeanresistors,americaninductors]{circuitikz}

\begin{document}
	
      \begin{circuitikz}[]

            \draw (0,0) to [short,o-](2,0)
            to [R=$R_1$] (2,-2) 
            to (1.5,-2)
            to [capacitor,C,  l_=$C$] (1.5,-4)
            to (2,-4)
            to (2,-5)
            to [short,-o] (0,-5);       
            \draw (2,-2) 
            to (2.5,-2) 
            to [R=$R_2$] (2.5,-4)
            to (2,-4);
	\end{circuitikz}
\end{document}
\end{center}
Сначала расчитаем импеданс параллельно соединенных конденсатора и резистора $R_2$
\begin{equation}
	\frac{1}{\hat{z_0}}=\frac{1}{R_2}+i\omega C
\end{equation}
\begin{equation}
	\hat{z_0}=\frac{R_2}{1+i \omega CR_2}
\end{equation}
Комплексный импеданс всей схемы будет равен:
\begin{equation}
	\hat{z}=\hat{z_0}+R_1=\frac{R_2}{1+i \omega R_2C}+R_1=
	\frac{R_2(1-i \omega R_2C)}{1+(\omega R_2C)^2}+R_1
\end{equation}
Отсюда
\begin{equation}
	\tan\phi = \frac{\Im\hat{z}}{\Re\hat{z}}=
	\frac{
		-\frac{\omega R_2^2C}{1+(\omega R_2C)^2}
	}{
		\frac{R_2+R_1+R_1(\omega R_2C)^2}{1+(\omega R_2C)^2}
	}=
	\frac{
		-\omega R_2^2C
	}{
		R_2+R_1+R_1(\omega R_2C)^2
	}	
\end{equation}


\section{Четвертая схема}
	\end{document}


