\input{text/diss}

\begin{document}

\def\labauthors{Понур К.А., Сарафанов Ф.Г., Сидоров Д.А.}
\def\labgroup{420}
\def\labnumber{210}
\def\labtheme{Исследование линейных двухполюсников и четырёхполюсников}
\renewcommand{\vec}{\mathbf}
\renewcommand{\Re}{\operatorname{Re}}
\renewcommand{\Im}{\operatorname{Im}}

\input{text/titlepage}

\tableofcontents
\newpage
\section{Схема №1. Последовательная $RC$ -- цепочка}
\begin{center}
	\input{chems/chem1}
\end{center}

Рассчитаем импеданс конденсатора методом комплексных амплитуд.
\begin{equation}
	\hat{U}=U_0 e^{i(\omega t+\phi_U)}
\end{equation}
Величину $\hat{U_0}=U_0e^{i\phi_U}$ будем называть комплексной амплитудой напряжения
\begin{equation}
	I=C\diff{U}{t}
\end{equation}
Отсюда получаем:
\begin{equation}
	\hat{I}=U_0\,\omega i C\exp(i\omega t+\phi_U)
\end{equation}
И комплексная амплитуда тока:
\begin{equation}
	\hat{I_0}=U_0i\omega C e^{i\phi_U}
\end{equation}
Получаем комплексный импеданс конденсатора
\begin{equation}
	\hat{z}_C=\frac{\hat{U_0}}{\hat{I_0}}=\frac{U_0e^{i\phi_U}}{U_0i\omega C e^{i\phi_U}}=\frac{1}{i\cdot\omega C}
\end{equation}
Нетрудно получить, что $\hat{z}_R=R$. Тогда импеданс $RC$ --цепочки
\begin{equation}
	\hat{z}=\frac{1}{i\cdot\omega C}+R
\end{equation}
\begin{equation}
	z=\sqrt{
		\frac{1}{\omega^2 C^2}	
		+R^2
	}=
	\sqrt{
		\frac{1}{\omega^2 C^2}	
		+\frac{R^2\omega^2 C^2}{\omega^2 C^2}
	}=\frac{\sqrt{1+(\omega RC)^2}}{\omega C}
\end{equation}
Экспериментально можно снимать зависомость $U_{13}\equiv U_\text{вх}$ и $U_{23}\equiv U_\text{вых}$ от частоты. Из закона Ома найдем тогда импеданс цепочки.
\begin{gather}
	\hat{J}_{13}=\hat{J}_{23} \quad\Rightarrow\quad
	\frac{\hat{U}_{13}}{\hat{z}}=\frac{\hat{U}_{23}}{R}
\end{gather}
Взяв по модулю получим нужное соотношение
\begin{equation}
	z=\frac{U_\text{вх}}{U_\text{вых}}R
\end{equation}
Также найдем зависимость разности фаз от частоты:
\begin{equation}
	|\tan\phi| = |\frac{\Im\hat{z}}{\Re\hat{z}}|=
	|\frac{
		-(\omega C)^{-1}
	}{
		R
	}|=
	\frac{
		1
	}{
		\omega RC
	}	
\end{equation}
\input{tables/table1_z}
\begin{figure}[H]
	\centering
	\includegraphics[width=0.85\textwidth]{img/chem1_z}
	\caption{Зависимость $z(\omega)$ для последовательной $RC$--цепочки}
	\label{fig:figure1}
\end{figure}
\begin{figure}[H]
	\centering
	\includegraphics[width=0.85\textwidth]{img/chem1_phi} 
	\caption{Зависимость $\tan\phi(\omega)$ для последовательной $RC$--цепочки}
	\label{fig:figure1}
\end{figure}
\section{Вторая схема}
\begin{center}
\documentclass[border=1pt]{standalone}
\usepackage[europeanresistors,americaninductors]{circuitikz}

\begin{document}
	
      \begin{circuitikz}[]
	\ctikzset {label/align = straight }
	\draw (0,0) coordinate (3) to [short,o-] (0,1)
	to [inductor, l^=$L$] (2,1) coordinate (3')
	to [R, l^=$R$] (4,1)
	to [short,-o] (4,0) coordinate (1)
	;
	\draw (3') to[short,-o] ++ (0,-1) coordinate (2);
	\draw (1) node [below, yshift=-1mm] {3};
	\draw (2) node [below, yshift=-1mm] {2};
	\draw (3) node [below, yshift=-1mm] {1};
	\end{circuitikz}
\end{document}
\end{center}

\begin{equation}
	\hat{I}=I_0e^{i(\omega t+\phi_I)}
\end{equation}
\begin{equation}
	U=L\diff{I}{t}
\end{equation}
\begin{equation}
	\hat{U}=I_0i\omega L e^{i(\omega t +\phi_I)}
\end{equation}
Отсюда
\begin{equation}
	\hat{z}=i\omega L
	\end{equation}

\begin{figure}[H]
	\centering
	\includegraphics[width=0.85\textwidth]{img/chem2_z}
	\caption{Зависимость $z(\omega)$ для последовательной $LC$--цепочки}
	\label{fig:figure1}
\end{figure}
\begin{figure}[H]
	\centering
	\includegraphics[width=0.85\textwidth]{img/chem2_phi} 
	\caption{Зависимость $\tan\phi(\omega)$ для последовательной $LC$--цепочки}
	\label{fig:figure1}
\end{figure}

\section{Третья схема}
\begin{center}
\documentclass[border=1pt]{standalone}
\usepackage[europeanresistors,americaninductors]{circuitikz}

\begin{document}
	
      \begin{circuitikz}[]

            \draw (0,0) to [short,o-](2,0)
            to [R=$R_1$] (2,-2) 
            to (1.5,-2)
            to [capacitor,C,  l_=$C$] (1.5,-4)
            to (2,-4)
            to (2,-5)
            to [short,-o] (0,-5);       
            \draw (2,-2) 
            to (2.5,-2) 
            to [R=$R_2$] (2.5,-4)
            to (2,-4);
	\end{circuitikz}
\end{document}
\end{center}
Сначала расчитаем импеданс параллельно соединенных конденсатора и резистора $R_2$
\begin{equation}
	\frac{1}{\hat{z_0}}=\frac{1}{R_2}+i\omega C
\end{equation}
\begin{equation}
	\hat{z_0}=\frac{R_2}{1+i \omega CR_2}
\end{equation}
Комплексный импеданс всей схемы будет равен:
\begin{equation}
	\hat{z}=\hat{z_0}+R_1=\frac{R_2}{1+i \omega R_2C}+R_1=
	\frac{R_2(1-i \omega R_2C)}{1+(\omega R_2C)^2}+R_1
\end{equation}
Отсюда
\begin{equation}
	\tan\phi = \frac{\Im\hat{z}}{\Re\hat{z}}=
	\frac{
		-\frac{\omega R_2^2C}{1+(\omega R_2C)^2}
	}{
		\frac{R_2+R_1+R_1(\omega R_2C)^2}{1+(\omega R_2C)^2}
	}=
	\frac{
		-\omega R_2^2C
	}{
		R_2+R_1+R_1(\omega R_2C)^2
	}	
\end{equation}


\section{Четвертая схема}
\begin{center}
\documentclass[border=1pt]{standalone}
\usepackage[europeanresistors,americaninductors]{circuitikz}

\begin{document}
	
      \begin{circuitikz}[]

            \draw (0,0) to (2,0)
            to [R=$R_1$] (2,-2) 
            to (1.5,-2)
            to [L] (1.5,-4)
            to (2,-4)
            to (2,-5)
            to (0,-5);       
            \draw (2,-2) 
            to (2.5,-2) 
            to [R=$R_2$] (2.5,-4)
            to (2,-4);
	\end{circuitikz}
\end{document}
\end{center}
Рассчитаем импеданс параллельно соединенных катушки и резистора $R_2$
\begin{equation}
	\frac{1}{\hat{z_1}}=\frac{1}{R_2}+\frac{1}{i\omega L}
\end{equation}
\begin{equation}
	\hat{z_1}=\frac{R_2\omega^2L^2+iR_2^2\omega L}{R_2+\omega L}
\end{equation}
А импеданс всей схемы:
\begin{equation}
	\hat{z_0}=\frac{R_2\omega^2L^2}{R_2+\omega L} + R_1+i\frac{iR_2^2\omega L}{R_2+\omega L}
\end{equation}


\section{Пятая схема}
\begin{center}
\documentclass[border=1pt]{standalone}
\usepackage[europeanresistors,americaninductors]{circuitikz}

\begin{document}
	
      \begin{circuitikz}[]
      \ctikzset {label/align = straight }
            \draw (0,0) to [short,o-](0,3)
            to (3,3)
            to (3,2);
            \draw (0,-1) to [short,o-](0,-4) 
            to (3,-4) 
            to (3,-3);


            \draw (3,2) to [capacitor,l_=$C_1$](0.5,-0.5) %левая ветка
            to [R,l_=$R_1$](3,-3);
            \draw (3,2) to [R, l^=$R_2$] (5.5,-0.5) %правая ветка
            to [capacitor,l^=$C_2$](3,-3);
            \draw (0.5,-0.5) to [short,-o] (2,-0.5);
            \draw   (5.5,-0.5) to [short,-o](4,-0.5);
            	\end{circuitikz}
\end{document}
\end{center}



\section{Шестая схема}
\begin{center}
\documentclass[border=1pt]{standalone}
\usepackage[europeanresistors,americaninductors]{circuitikz}
\tikzset{
  pics/carc/.style args={#1:#2:#3}{
    code={
      \draw[pic actions] (#1:#3) arc(#1:#2:#3);
    }
  }
}

\begin{document}
	
      \begin{circuitikz}[]
      \draw (0,0) to [short,o-*,C=$C$](4,0)
      to [short,*-,C=$C$] ++(4,0)
      to [short,*-,C=$C$] ++(4,0)
      to  [short,-o] ++(2,0);

      \draw (0,-2) node  {$U_{in}$};
      \draw (14,-2) node  {$U_{out}$};

      \draw[thick] (2,-2) pic[red, -latex]{carc=-150:150:1.3cm} node {$J_1$};
      \draw[thick] (6,-2) pic[red, -latex]{carc=-150:150:1.3cm} node {$J_2$};
      \draw[thick] (10,-2) pic[red, -latex]{carc=-150:150:1.3cm} node {$J_3$};

      \draw (0,-4) to [short,o-](4,-4)
      to  [short,*-](8,-4)
      to  [short,*-](12,-4)
      to  [short,-o] (14,-4);

      \draw (4,0) to [R=$R$](4,-4);
      \draw (8,0) to [R=$R$](8,-4);
      \draw (12,0) to [short,*-*,R=$R$](12,-4);

      \end{circuitikz}
\end{document}
\end{center}


	\end{document}


