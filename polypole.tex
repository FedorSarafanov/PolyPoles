\input{text/diss}

\begin{document}

\def\labauthors{Понур К.А., Сарафанов Ф.Г., Сидоров Д.А.}
\def\labgroup{420}
\def\labnumber{210}
\def\labtheme{Исследование линейных двухполюсников и четырёхполюсников}
\renewcommand{\vec}{\mathbf}
\renewcommand{\Re}{\operatorname{Re}}
\renewcommand{\Im}{\operatorname{Im}}
\renewcommand{\phi}{\varphi}
\renewcommand{\hat}{\widehat}

\input{text/titlepage}

\tableofcontents
\newpage

\section{Расчет цепей}
\subsection{Расчет импеданса некоторых линейных элементов}

	Будем рассчитывать импеданс методом комплексных амплитуд.
	Полагая известным
	\begin{equation}
		\hat{U}=
		U_0 e^{i(\omega t+\phi_u)}=
		U_0\exp(i\phi_u)\exp(i\omega t)=
		\hat{U}_0\exp(i\omega t)
	\end{equation}
	где	$\hat{U}_0=U_0\exp(i\phi_u)$ -- комплексная амплитуд напряжения, включающая в себя начальную фазу.
	
	Будем предполагать, что мы нашли $\hat{J}=\hat{J}(\hat{U})$, используя связь тока и напряжения:
	\begin{equation}
		\hat{J}=\hat{J}_0\exp(i\omega t)
	\end{equation}	

	Возможен обратный ход -- от известного тока через линейную связь перейти к напряжению.

	Тогда импеданс по определению найдется как
	\begin{equation}
		\hat{z}=\frac{\hat{U}_0}{\hat{J}_0}
	\end{equation}
\subsubsection{Импеданс конденсатора}
	Рассчитаем импеданс конденсатора методом комплексных амплитуд.
	\begin{equation}
		\hat{J}=C\diff{\hat{U}}{t}
	\end{equation}
	Отсюда получаем:
	\begin{equation}
		\hat{J}=i\omega C U_0\exp(i\phi_u)\exp(i\omega t)
	\end{equation}
	И комплексная амплитуда тока:
	\begin{equation}
		\hat{J}_0=i\omega C U_0\exp(i\phi_u)
	\end{equation}
	Получаем комплексный импеданс конденсатора
	\begin{equation}
		\hat{z}_C=\frac{\hat{U}_0}{\hat{J}_0}=\frac{U_0\exp(i\phi_u)}{U_0i\omega C \exp(i\phi_u)}=\frac{1}{i\cdot\omega C}
	\end{equation}

\subsubsection{Импеданс индуктивности}	

	В данном случае удобно считать известным ток.
	\begin{equation}
		\hat{U}=L\diff{\hat{J}}{t}
	\end{equation}
	Отсюда получаем:
	\begin{equation}
		\hat{U}=i\omega L J_0\exp(i\phi_j)\exp(i\omega t)
	\end{equation}
	И комплексная амплитуда напряжения:
	\begin{equation}
		\hat{U}_0=i\omega L J_0\exp(i\phi_j)
	\end{equation}
	Получаем комплексный импеданс конденсатора
	\begin{equation}
		\hat{z}_L=\frac{\hat{U}_0}{\hat{J}_0}=\frac{i\omega L J_0\exp(i\phi_j)}{J_0\exp(i\phi_j)}=i\cdot \omega L
	\end{equation}

\subsubsection{Импеданс резистора}	

	Пусть известен ток. 
	\begin{equation}
		\hat{U}=\hat{J}R
	\end{equation}
	Отсюда получаем:
	\begin{equation}
		\hat{U}=R J_0\exp(i\phi_j)\exp(i\omega t)
	\end{equation}
	И комплексная амплитуда напряжения:
	\begin{equation}
		\hat{U}_0=R J_0\exp(i\phi_j)
	\end{equation}
	Получаем комплексный импеданс конденсатора
	\begin{equation}
		\hat{z}_R=\frac{\hat{U}_0}{\hat{J}_0}=\frac{R J_0\exp(i\phi_j)}{J_0\exp(i\phi_j)}=R
	\end{equation}

\section{Двухполюсники. Расчет цепи и экспериментальные данные}
\subsection{Схема №1. Последовательная $RC$ -- цепочка}

\begin{figure}[H]
	\centering
	\includegraphics[]{chems/chem1}
	\caption{Последовательная $RC$ -- цепочка}
	\label{fig:figure1}
\end{figure}

\subsubsection{Импеданс}
    
Импеданс $RC$ -- цепочки найдем, используя ранее вычисленные 
импедансы линейных элементов:
\begin{equation}
	\hat{z}=\frac{1}{i\cdot\omega C}+R
\end{equation}
\begin{equation}
	z=\sqrt{
		\frac{1}{\omega^2 C^2}	
		+R^2
	}=
	\sqrt{
		\frac{1}{\omega^2 C^2}	
		+\frac{R^2\omega^2 C^2}{\omega^2 C^2}
	}=\frac{\sqrt{1+(\omega RC)^2}}{\omega C}
\end{equation}
Экспериментально можно снимать зависомость $U_{13}\equiv U_\text{вх}$ и $U_{23}\equiv U_\text{вых}$ от частоты. Из закона Ома найдем тогда импеданс цепочки.
\begin{gather}
	\hat{J}_{13}=\hat{J}_{23} \quad\Rightarrow\quad
	\frac{\hat{U}_{13}}{\hat{z}}=\frac{\hat{U}_{23}}{R}
\end{gather}
Взяв по модулю, получим нужное соотношение:
\begin{equation}
	z=\frac{U_\text{вх}}{U_\text{вых}}R
\end{equation}

\subsubsection{Разность фаз}
Также найдем зависимость разности фаз от частоты:
\begin{equation}
	|\tan\phi| = |\frac{\Im\hat{z}}{\Re\hat{z}}|=
	|\frac{
		-(\omega C)^{-1}
	}{
		R
	}|=
	\frac{
		1
	}{
		\omega RC
	}	
\end{equation}

\subsubsection{Результаты эксперимента}
\input{tables/table1_z}
\begin{figure}[H]
	\centering
	\includegraphics[width=0.85\textwidth]{img/chem1_z}
	\caption{Зависимость $z(\omega)$ для последовательной $RC$--цепочки}
	\label{fig:figure1}
\end{figure}
\begin{figure}[H]
	\centering
	\includegraphics[width=0.85\textwidth]{img/chem1_phi} 
	\caption{Зависимость $\tan\phi(\omega)$ для последовательной $RC$--цепочки}
	\label{fig:figure1}
\end{figure}


\subsection{Схема №2. Последовательная $LC$ -- цепочка}
\begin{figure}[H]
	\centering
	\includegraphics[]{chems/chem1}
	\caption{Последовательная $RC$ -- цепочка}
	\label{fig:figure1}
\end{figure}

\subsubsection{Импеданс}
\begin{equation}
	\hat{z}=i\omega L+R
\end{equation}
\begin{equation}
	z=\sqrt{(\omega L)^2+R}
\end{equation}
Очевидно, что аналогично последовательной $RC$--цепочке
\begin{equation}
	z=\frac{U_\text{вх}}{U_\text{вых}}R
\end{equation}
\subsubsection{Разность фаз}
\begin{equation}
	\left|\tan\phi\right| = \left|\frac{\Im\hat{z}}{\Re\hat{z}}\right|=
	\left|\frac{
		\omega L
	}{
		R
	}\right|=
	\frac{
		\omega L
	}{
		R
	}	
\end{equation}

\subsubsection{Результаты эксперимента}

\input{tables/table2_z}
\begin{figure}[H]
	\centering
	\includegraphics[width=0.85\textwidth]{img/chem2_z}
	\caption{Зависимость $z(\omega)$ для последовательной $LC$--цепочки}
	\label{fig:figure1}
\end{figure}
\begin{figure}[H]
	\centering
	\includegraphics[width=0.85\textwidth]{img/chem2_phi} 
	\caption{Зависимость $\tan\phi(\omega)$ для последовательной $LC$--цепочки}
	\label{fig:figure1}
\end{figure}

\subsection{Схема №3. Двухполюсник $R[RC]$}
\begin{figure}[H]
	\centering
	\includegraphics[]{chems/chem3}
	\caption{Двухполюсник $R[RC]$}
	\label{fig:RRC}
\end{figure}

\subsubsection{Импеданс}
Сначала рассчитаем импеданс параллельно соединенных конденсатора и резистора $R$
\begin{equation}
	\frac{1}{\hat{z}_0}=\frac{1}{R}+i\omega C
\end{equation}
\begin{equation}
	\hat{z}_0=\frac{R}{1+i \omega CR}
\end{equation}
Комплексный импеданс всей схемы будет равен:
\begin{equation}
	\hat{z}=\hat{z}_0+R=\frac{R}{1+i \omega RC}+R=
	\frac{R(1-i \omega RC)}{1+(\omega RC)^2}+R
\end{equation}
\begin{equation}
	z=\sqrt{\Im^2\hat{z}+\Re^2\hat{z}}=
	R\sqrt{
	\left(
	1+\frac{1}{1+(\omega RC)^2}
	\right)^2+
	\left(
	\frac{\omega C}{1+(\omega RC)^2}
	\right)^2
	}
\end{equation}

\subsubsection{Разность фаз}
\begin{equation}
	\tan\phi = \frac{\Im\hat{z}}{\Re\hat{z}}=
	\frac{
		-\frac{\omega R^2C}{1+(\omega RC)^2}
	}{
		\frac{R+R+R(\omega RC)^2}{1+(\omega RC)^2}
	}=
	\frac{
		-\omega R^2C
	}{
		R+R+R(\omega RC)^2
	}=
	-\frac{
		\omega RC
	}{
		2+(\omega RC)^2
	}	
\end{equation}
Из уравнения видно, что на малых частотах $z\approx 2R$, а при высоких $z\approx R$.

\subsubsection{Результаты эксперимента}

\input{tables/table3_z}
\begin{figure}[H]
	\centering
	\includegraphics[width=0.85\textwidth]{img/chem3_z}
	\caption{Зависимость $z(\omega)$ для  $R[RC]$--двухполюсника}
	\label{fig:figure1}
\end{figure}
\begin{figure}[H]
	\centering
	\includegraphics[width=0.85\textwidth]{img/chem3_phi} 
	\caption{Зависимость $\tan\phi(\omega)$ для $R[RC]$--двухполюсника}
	\label{fig:figure1}
\end{figure}


\subsection{Четвертая схема}
\begin{center}
\documentclass[border=1pt]{standalone}
\usepackage[europeanresistors,americaninductors]{circuitikz}

\begin{document}
	
      \begin{circuitikz}[]

            \draw (0,0) to (2,0)
            to [R=$R_1$] (2,-2) 
            to (1.5,-2)
            to [L] (1.5,-4)
            to (2,-4)
            to (2,-5)
            to (0,-5);       
            \draw (2,-2) 
            to (2.5,-2) 
            to [R=$R_2$] (2.5,-4)
            to (2,-4);
	\end{circuitikz}
\end{document}
\end{center}
Рассчитаем импеданс параллельно соединенных катушки и резистора $R$
\begin{equation}
	\frac{1}{\hat{z_1}}=\frac{1}{R}+\frac{1}{i\omega L}
\end{equation}
\begin{equation}
	\hat{z_1}=\frac{R\omega^2L^2+iR^2\omega L}{R+\omega L}
\end{equation}
А импеданс всей схемы:
\begin{equation}
	\hat{z}_0=\frac{R\omega^2L^2}{R+\omega L} + R+i\frac{iR^2\omega L}{R+\omega L}
\end{equation}


\section{Пятая схема}
\begin{center}
\documentclass[border=1pt]{standalone}
\usepackage[europeanresistors,americaninductors]{circuitikz}

\begin{document}
	
      \begin{circuitikz}[]
      \ctikzset {label/align = straight }
            \draw (0,0) to [short,o-](0,3)
            to (3,3)
            to (3,2);
            \draw (0,-1) to [short,o-](0,-4) 
            to (3,-4) 
            to (3,-3);


            \draw (3,2) to [capacitor,l_=$C_1$](0.5,-0.5) %левая ветка
            to [R,l_=$R_1$](3,-3);
            \draw (3,2) to [R, l^=$R_2$] (5.5,-0.5) %правая ветка
            to [capacitor,l^=$C_2$](3,-3);
            \draw (0.5,-0.5) to [short,-o] (2,-0.5);
            \draw   (5.5,-0.5) to [short,-o](4,-0.5);
            	\end{circuitikz}
\end{document}
\end{center}



\section{Шестая схема}
\begin{center}
\documentclass[border=1pt]{standalone}
\usepackage[europeanresistors,americaninductors]{circuitikz}
\tikzset{
  pics/carc/.style args={#1:#2:#3}{
    code={
      \draw[pic actions] (#1:#3) arc(#1:#2:#3);
    }
  }
}

\begin{document}
	
      \begin{circuitikz}[]
      \draw (0,0) to [short,o-*,C=$C$](4,0)
      to [short,*-,C=$C$] ++(4,0)
      to [short,*-,C=$C$] ++(4,0)
      to  [short,-o] ++(2,0);

      \draw (0,-2) node  {$U_{in}$};
      \draw (14,-2) node  {$U_{out}$};

      \draw[thick] (2,-2) pic[red, -latex]{carc=-150:150:1.3cm} node {$J_1$};
      \draw[thick] (6,-2) pic[red, -latex]{carc=-150:150:1.3cm} node {$J_2$};
      \draw[thick] (10,-2) pic[red, -latex]{carc=-150:150:1.3cm} node {$J_3$};

      \draw (0,-4) to [short,o-](4,-4)
      to  [short,*-](8,-4)
      to  [short,*-](12,-4)
      to  [short,-o] (14,-4);

      \draw (4,0) to [R=$R$](4,-4);
      \draw (8,0) to [R=$R$](8,-4);
      \draw (12,0) to [short,*-*,R=$R$](12,-4);

      \end{circuitikz}
\end{document}
\end{center}


	\end{document}


