\input{text/diss}

\begin{document}

\def\labauthors{Понур К.А., Сарафанов Ф.Г., Сидоров Д.А.}
\def\labgroup{420}
\def\labnumber{210}
\def\labtheme{Исследование линейных двухполюсников и четырёхполюсников}
\renewcommand{\vec}{\mathbf}
\renewcommand{\Re}{\operatorname{Re}}
\renewcommand{\Im}{\operatorname{Im}}
\renewcommand{\phi}{\varphi}
\renewcommand{\hat}{\widehat}

\input{text/titlepage}

\tableofcontents
\newpage

\section{Расчет цепей}
\subsection{Расчет импеданса некоторых линейных элементов}

	Будем рассчитывать импеданс методом комплексных амплитуд.
	Полагая известным
	\begin{equation}
		\hat{U}=
		U_0 e^{i(\omega t+\phi_u)}=
		U_0\exp(i\phi_u)\exp(i\omega t)=
		\hat{U}_0\exp(i\omega t)
	\end{equation}
	где	$\hat{U}_0=U_0\exp(i\phi_u)$ -- комплексная амплитуд напряжения, включающая в себя начальную фазу.
	
	Будем предполагать, что мы нашли $\hat{J}=\hat{J}(\hat{U})$, используя связь тока и напряжения:
	\begin{equation}
		\hat{J}=\hat{J}_0\exp(i\omega t)
	\end{equation}	

	Возможен обратный ход -- от известного тока через линейную связь перейти к напряжению.

	Тогда импеданс по определению найдется как
	\begin{equation}
		\hat{z}=\frac{\hat{U}_0}{\hat{J}_0}
	\end{equation}
\subsubsection{Импеданс конденсатора}
	Рассчитаем импеданс конденсатора методом комплексных амплитуд.
	\begin{equation}
		\hat{J}=C\diff{\hat{U}}{t}
	\end{equation}
	Отсюда получаем:
	\begin{equation}
		\hat{J}=i\omega C U_0\exp(i\phi_u)\exp(i\omega t)
	\end{equation}
	И комплексная амплитуда тока:
	\begin{equation}
		\hat{J}_0=i\omega C U_0\exp(i\phi_u)
	\end{equation}
	Получаем комплексный импеданс конденсатора
	\begin{equation}
		\hat{z}_C=\frac{\hat{U}_0}{\hat{J}_0}=\frac{U_0\exp(i\phi_u)}{U_0i\omega C \exp(i\phi_u)}=\frac{1}{i\cdot\omega C}
	\end{equation}

\subsubsection{Импеданс индуктивности}	

	В данном случае удобно считать известным ток.
	\begin{equation}
		\hat{U}=L\diff{\hat{J}}{t}
	\end{equation}
	Отсюда получаем:
	\begin{equation}
		\hat{U}=i\omega L J_0\exp(i\phi_j)\exp(i\omega t)
	\end{equation}
	И комплексная амплитуда напряжения:
	\begin{equation}
		\hat{U}_0=i\omega L J_0\exp(i\phi_j)
	\end{equation}
	Получаем комплексный импеданс конденсатора
	\begin{equation}
		\hat{z}_L=\frac{\hat{U}_0}{\hat{J}_0}=\frac{i\omega L J_0\exp(i\phi_j)}{J_0\exp(i\phi_j)}=i\cdot \omega L
	\end{equation}

\subsubsection{Импеданс резистора}	

	Пусть известен ток. 
	\begin{equation}
		\hat{U}=\hat{J}R
	\end{equation}
	Отсюда получаем:
	\begin{equation}
		\hat{U}=R J_0\exp(i\phi_j)\exp(i\omega t)
	\end{equation}
	И комплексная амплитуда напряжения:
	\begin{equation}
		\hat{U}_0=R J_0\exp(i\phi_j)
	\end{equation}
	Получаем комплексный импеданс конденсатора
	\begin{equation}
		\hat{z}_R=\frac{\hat{U}_0}{\hat{J}_0}=\frac{R J_0\exp(i\phi_j)}{J_0\exp(i\phi_j)}=R
	\end{equation}

\section{Двухполюсники. Расчет цепи и экспериментальные данные}
\subsection{Схема №1. Последовательная $RC$ -- цепочка}

\begin{figure}[H]
	\centering
	\includegraphics[]{chems/chem1}
	\caption{Последовательная $RC$ -- цепочка}
	\label{fig:RC}
\end{figure}

\subsubsection{Импеданс}
    
Импеданс $RC$ -- цепочки найдем, используя ранее вычисленные 
импедансы линейных элементов:
\begin{equation}
	\hat{z}=\frac{1}{i\cdot\omega C}+R
\end{equation}
\begin{equation}
	z=\sqrt{
		\frac{1}{\omega^2 C^2}	
		+R^2
	}=
	\sqrt{
		\frac{1}{\omega^2 C^2}	
		+\frac{R^2\omega^2 C^2}{\omega^2 C^2}
	}=\frac{\sqrt{1+(\omega RC)^2}}{\omega C}
\end{equation}
Экспериментально можно снимать зависомость $U_{13}\equiv U_\text{вх}$ и $U_{23}\equiv U_\text{вых}$ от частоты. Из закона Ома найдем тогда импеданс цепочки.
\begin{gather}
	\hat{J}_{13}=\hat{J}_{23} \quad\Rightarrow\quad
	\frac{\hat{U}_{13}}{\hat{z}}=\frac{\hat{U}_{23}}{R}
\end{gather}
Взяв по модулю, получим нужное соотношение:
\begin{equation}
	z=\frac{U_\text{вх}}{U_\text{вых}}R
\end{equation}

\subsubsection{Разность фаз}
Также найдем зависимость разности фаз от частоты:
\begin{equation}
	|\tan\phi| = |\frac{\Im\hat{z}}{\Re\hat{z}}|=
	|\frac{
		-(\omega C)^{-1}
	}{
		R
	}|=
	\frac{
		1
	}{
		\omega RC
	}	
\end{equation}

\subsubsection{Результаты эксперимента}
\input{tables/table1_z}
\begin{figure}[H]
	\centering
	\includegraphics[width=0.85\textwidth]{img/chem1_z}
	\caption{Зависимость $z(\omega)$ для последовательной $RC$--цепочки}
	\label{fig:RC_z}
\end{figure}
\begin{figure}[H]
	\centering
	\includegraphics[width=0.85\textwidth]{img/chem1_phi} 
	\caption{Зависимость $\tan\phi(\omega)$ для последовательной $RC$--цепочки}
	\label{fig:RC_tanphi}
\end{figure}


\subsection{Схема №2. Последовательная $LC$ -- цепочка}
\begin{figure}[H]
	\centering
	\includegraphics[]{chems/chem1}
	\caption{Последовательная $LC$ -- цепочка}
	\label{fig:LC}
\end{figure}

\subsubsection{Импеданс}
\begin{equation}
	\hat{z}=i\omega L+R
\end{equation}
\begin{equation}
	z=\sqrt{(\omega L)^2+R}
\end{equation}
Очевидно, что аналогично последовательной $RC$--цепочке
\begin{equation}
	z=\frac{U_\text{вх}}{U_\text{вых}}R
\end{equation}
\subsubsection{Разность фаз}
\begin{equation}
	\left|\tan\phi\right| = \left|\frac{\Im\hat{z}}{\Re\hat{z}}\right|=
	\left|\frac{
		\omega L
	}{
		R
	}\right|=
	\frac{
		\omega L
	}{
		R
	}	
\end{equation}

\subsubsection{Результаты эксперимента}

\input{tables/table2_z}
\begin{figure}[H]
	\centering
	\includegraphics[width=0.85\textwidth]{img/chem2_z}
	\caption{Зависимость $z(\omega)$ для последовательной $LC$--цепочки}
	\label{fig:LC_z}
\end{figure}
\begin{figure}[H]
	\centering
	\includegraphics[width=0.85\textwidth]{img/chem2_phi} 
	\caption{Зависимость $\tan\phi(\omega)$ для последовательной $LC$--цепочки}
	\label{fig:LC_tanphi}
\end{figure}

\subsection{Схема №3. Двухполюсник $R[RC]$}
\begin{figure}[H]
	\centering
	\includegraphics[]{chems/chem3}
	\caption{Двухполюсник $R[RC]$}
	\label{fig:RRC}
\end{figure}

\subsubsection{Импеданс}
Сначала рассчитаем импеданс параллельно соединенных конденсатора и резистора $R$
\begin{equation}
	\frac{1}{\hat{z}_0}=\frac{1}{R}+i\omega C
\end{equation}
\begin{equation}
	\hat{z}_0=\frac{R}{1+i \omega CR}
\end{equation}
Комплексный импеданс всей схемы будет равен:
\begin{equation}
	\hat{z}=\hat{z}_0+R=\frac{R}{1+i \omega RC}+R=
	\frac{R(1-i \omega RC)}{1+(\omega RC)^2}+R
\end{equation}
\begin{equation}
	z=\sqrt{\Im^2\hat{z}+\Re^2\hat{z}}=
	R\sqrt{
	\left(
	1+\frac{1}{1+(\omega RC)^2}
	\right)^2+
	\left(
	\frac{\omega C}{1+(\omega RC)^2}
	\right)^2
	}
\end{equation}

\subsubsection{Разность фаз}
\begin{equation}
	\tan\phi = \frac{\Im\hat{z}}{\Re\hat{z}}=
	\frac{
		-\frac{\omega R^2C}{1+(\omega RC)^2}
	}{
		\frac{R+R+R(\omega RC)^2}{1+(\omega RC)^2}
	}=
	\frac{
		-\omega R^2C
	}{
		R+R+R(\omega RC)^2
	}=
	-\frac{
		\omega RC
	}{
		2+(\omega RC)^2
	}	
\end{equation}
Из уравнения видно, что на малых частотах $z\approx 2R$, а при высоких $z\approx R$.

\subsubsection{Результаты эксперимента}

\input{tables/table3_z}
\begin{figure}[H]
	\centering
	\includegraphics[width=0.85\textwidth]{img/chem3_z}
	\caption{Зависимость $z(\omega)$ для  $R[RC]$--двухполюсника}
	\label{fig:RRC_z}
\end{figure}
\begin{figure}[H]
	\centering
	\includegraphics[width=0.85\textwidth]{img/chem3_phi} 
	\caption{Зависимость $\tan\phi(\omega)$ для $R[RC]$--двухполюсника}
	\label{fig:RRC_tanphi}
\end{figure}


\subsection{Схема №4. Двухполюсник $R[RL]$}
\begin{figure}[H]
	\centering
	\includegraphics[]{chems/chem4}
	\caption{Двухполюсник $R[RL]$}
	\label{fig:RRL}
\end{figure}
\subsubsection{Импеданс}
Рассчитаем импеданс параллельно соединенных катушки и резистора $R$
\begin{equation}
	\frac{1}{\hat{z_0}}=\frac{1}{R}+\frac{1}{i\omega L}=\frac{R+i\omega L}{iR\omega L}=\frac{\omega L - iR}{\omega R L}
\end{equation}
\begin{equation}
	\hat{z_0}=\frac{\omega R L(\omega L + iR)}{(\omega L - iR)(\omega L + iR)}=
	\frac{\omega^2L^2 R +i\omega L R^2}{\omega^2L^2+R^2}
\end{equation}
А импеданс всей схемы:
\begin{equation}
	\hat{z}=
	\left(
	\frac{2\omega^2L^2 R+R^3}{\omega^2L^2+R^2}
	\right)+
	i
	\left(
	\frac{\omega L R^2}{\omega^2L^2+R^2}
	\right)
\end{equation}
\begin{equation}
	z=\sqrt{\Im^2\hat{z}+\Re^2\hat{z}}=
	\frac{1}{\omega^2L^2+R^2}\sqrt{
	\left(
	2\omega^2L^2 R+R^3
	\right)^2+
	\left(
	\omega L R^2
	\right)^2
	}
\end{equation}
При больших частотах можно пренебречь вторым слагаемым под корнем и сопротивлением в суммах, тогда видно, что на таких частотах $z\approx 2R$.

На малых частотах $\omega\approx0$ предел даёт значение импеданса $R$.
\subsubsection{Разность фаз}
\begin{equation}
	\tan\phi = \frac{\Im\hat{z}}{\Re\hat{z}}=
	\frac{
		\omega L R^2
	}{
		2\omega^2L^2 R+R^3
	}=
	\frac{
		\omega LR
	}{
		2\omega^2L^2 +R^2
	}	
\end{equation}

\subsubsection{Результаты эксперимента}

\input{tables/table4_z}
\begin{figure}[H]
	\centering
	\includegraphics[width=0.85\textwidth]{img/chem4_z}
	\caption{Зависимость $z(\omega)$ для  $R[RL]$--двухполюсника}
	\label{fig:RRL_z}
\end{figure}
\begin{figure}[H]
	\centering
	\includegraphics[width=0.85\textwidth]{img/chem4_phi} 
	\caption{Зависимость $\tan\phi(\omega)$ для $R[RL]$--двухполюсника}
	\label{fig:RRL_tanphi}
\end{figure}


\section{Четырехполюсники. Расчет цепи и экспериментальные данные}

\subsection{Схема №5. Простейший мостовой фазовращатель}

\begin{figure}[H]
	\centering
	\includegraphics[]{chems/chem5}
	\caption{Принципиальная схема фазовращателя}
	\label{fig:ph_rot}
\end{figure}

Запишем ток через левую ветвь:
\begin{equation}
	\hat{J}_{left}\equiv \hat{J}_{R_1}=\frac{\hat{U}_{in}}{R_1+\frac{1}{i\omega C_1}}
\end{equation}
Аналогично через правую:
\begin{equation}
	\hat{J}_{right}\equiv \hat{J}_{R_2}=\frac{\hat{U}_{in}}{R_2+\frac{1}{i\omega C_2}}
\end{equation}
Тогда 
\begin{equation}
	\hat{U}_{out}\equiv\hat{U}_{MN}=
	-\hat{U}_{AM}+\hat{U}_{AN}=
	-\hat{J}_{left}\cdot\frac{1}{i\omega C_1}
	+\hat{J}_{right}\cdot R_2
\end{equation}
\begin{equation}
	\hat{U}_{out}=\hat{U}_{in}
	\left(
	\frac{R_2}{R_2+(i\omega C_2)^{-1}}-
	\frac{1}{i\omega C_1}\frac{1}{R_1+(i\omega C_1)^{-1}}
	\right)
\end{equation}
\begin{equation}
	\hat{K}=\frac{\hat{U}_{out}}{\hat{U}_{in}}=
	\frac{i\omega C_2 R_2}{i\omega C_2 R_2+1}-
	\frac{1}{i\omega C_1 R_1+1}
\end{equation}
Обозначим $\Omega_1=\omega C_1R_1$, $\Omega_2=\omega C_2R_2$:
\begin{equation}
	\hat{K}=\frac{i\Omega_2}{i\Omega_2+1}-\frac{1}{i\Omega_1+1}=
	\frac{i\Omega_2(1-i\Omega_2)}{\Omega_2^2+1}+\frac{i\Omega_1-1}{\Omega_1^2+1}
\end{equation}
\begin{equation}
	\hat{K}=
	\left(
		\frac{\Omega_2^2}{\Omega_2^2+1}-
		\frac{1}{\Omega_1^2+1}
	\right)+
	i\left(
		\frac{\Omega_2}{\Omega_2^2+1}+
		\frac{\Omega_1}{\Omega_1^2+1}
	\right)
\end{equation}
Отсюда
\begin{equation}
	K=
	\sqrt{
	\left(
		\frac{\Omega_2^2}{\Omega_2^2+1}-
		\frac{1}{\Omega_1^2+1}
	\right)^2+
	\left(
		\frac{\Omega_2}{\Omega_2^2+1}+
		\frac{\Omega_1}{\Omega_1^2+1}
	\right)^2
	}
\end{equation}
При $\Omega_1$=$\Omega_2$ подстановка дает $K\equiv 1$.

\begin{equation}
	\tan\phi=\left(
		\frac{\Omega_2}{\Omega_2^2+1}+
		\frac{\Omega_1}{\Omega_1^2+1}
	\right)
	\cdot
	\left(
		\frac{\Omega_2^2}{\Omega_2^2+1}-
		\frac{1}{\Omega_1^2+1}
	\right)^{-1}
\end{equation}
При $\Omega_1$=$\Omega_2 \equiv \Omega$
\begin{equation}
	\tan\phi=\frac{2\Omega}{1-\Omega^2}
\end{equation}
Можно заметить, что это формула тангенса половинного угла:
\begin{equation}
	\tan\phi=\frac{2\tan\frac{\phi}{2}}{1-\tan^2\frac{\phi}{2}}
\end{equation}

Отсюда
\begin{equation}
	\tan\frac{\phi}{2}=\Omega = \omega RC
\end{equation}

Так как $\arctan\Omega$ может принимать значения только от 0 до $\frac{\pi}2$, то 
\begin{equation}
	0\leq\phi<\pi
\end{equation}

\subsection{Схема №6. Составной четырехполюсник - фазовращатель}

\begin{figure}[H]
	\centering
	\includegraphics[]{chems/chem6}
	\caption{Принципиальная схема фазовращателя}
	\label{fig:ph_rot2}
\end{figure}

\begin{equation}
	z=\frac{1}{i\omega C}+
	\cfrac{1}{
		\cfrac{1}{R}+
		\cfrac{1}{
			\cfrac{1}{i\omega C}+
			\cfrac{1}{
				\frac{1}{R}+
					\cfrac{1}{
						R+
						\cfrac{1}{
							i\omega C				
						}
					}
				}
			}		
			}=
\frac{1}{i\omega C}+
	\cfrac{1}{
		\cfrac{1}{R}+
		\cfrac{1}{
			\cfrac{1}{i\omega C}+
			\cfrac{1}{
				\frac{1}{R}+
					\cfrac{i\omega C}{i\Omega+1}
				}
			}		
			}
\end{equation}
\begin{gather}
=
\frac{1}{i\omega C}+
	\cfrac{1}{
		\cfrac{1}{R}+
		\cfrac{1}{
			\cfrac{1}{i\omega C}+
			\cfrac{R(i\Omega+1)}{2i\Omega+1}	
		}	
	}
=\\=
\frac{1}{i\omega C}+
	\cfrac{1}{
		\cfrac{1}{R}+		
			\cfrac{i\omega C - 2\Omega \omega C}{3i\Omega-\Omega^2}
	}
=\\=
\frac{1}{i\omega C}+
	\cfrac{1}{
		\cfrac{3i\Omega-\Omega^2}{...}+		
			\cfrac{i\Omega - 2\Omega^2}{R(3i\Omega-\Omega^2)}
	}
=\\=
\frac{1}{i\omega C}+
\cfrac{R(3i\Omega-\Omega^2)}{4i\Omega - 3\Omega^2}
=\\=
\frac{4i\Omega - 3\Omega^2}{...}+
\cfrac{i\Omega(3i\Omega-\Omega^2)}{-4\Omega\omega C - 3i\Omega^2\omega C}
=\\=
\frac{4i\Omega - 3\Omega^2}{...}+
\cfrac{-3\Omega^2-i\Omega^3}{-4\Omega\omega C - 3i\Omega^2\omega C}
=\\=
R\frac{
	i\Omega^3
	+6\Omega^2
	-4i\Omega 
}{
	4\Omega^2 
	+3i\Omega^3
}
=\\=
R\frac{
	(i\Omega^3
	+6\Omega^2
	-4i\Omega )
	(
		4\Omega^2 
		-3i\Omega^3	
	)
}{
	16\Omega^4
	+9\Omega^6
}
=R\frac{
	-16 i \Omega^3 + 12 \Omega^4 - 14 i \Omega^5 + 3 \Omega^6
}{
	16\Omega^4
	+9\Omega^6
}
=\\=
\left(
R
	\frac{
		12 \Omega^4 + 3 \Omega^6
	}{
		16\Omega^4
		+9\Omega^6
	}
\right)
-i
\left(
	R\frac{
		16 \Omega^3 + 14 \Omega^5
	}{
		16\Omega^4
		+9\Omega^6
	}
\right)
\end{gather}

Будем рассматривать данный четырехполюсник как составной из трех следующего вида:
\begin{figure}[H]
	\centering
	\includegraphics[]{chems/chem6_one}
	\caption{Простой четырехполюсник}
	\label{fig:ph_rot2_one}
\end{figure}

Здесь найдем $\hat{K}_1$.
\begin{equation}
	\hat{U}_{out}\equiv\hat{U}_R=\hat{J}\cdot R=
	\frac{\hat{U}_{in}}{\hat{z}}R
\end{equation}
\begin{equation}
	\hat{K}_1=\frac{R}{\hat{z}}=\frac{R}{R+1/(i\omega C)}=\frac{i\Omega}{1+i\Omega}=
	\frac{\Omega^2}{1+\Omega^2}+i\frac{\Omega}{1+\Omega^2}
\end{equation}
Здесь
\begin{equation}
	K_1=\frac{\Omega}{\sqrt{1+\Omega^2}}
\end{equation}
\begin{equation}
	\tan\phi=\frac{1}{\Omega}
\end{equation}
Очевидно, что при соединении трех таких четырехполюсников 
\begin{equation}
	\hat{K}=\hat{K}_1^3=K_1^3\exp(3i\phi_1)
\end{equation}

\end{document}